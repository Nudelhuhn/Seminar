\documentclass[conference]{IEEEtran}
\usepackage[utf8]{inputenc}
\usepackage{url}
\usepackage{amsmath}
\usepackage{amssymb}
\usepackage[colorlinks=true, linkcolor=blue, citecolor=blue, urlcolor=blue]{hyperref}


\begin{document}

\title{Seminar - Collective navigation of a multi-robot system in an unknown environment}
\author{\IEEEauthorblockN{Gregor Germerodt}
\IEEEauthorblockA{Hochschule Trier - Trier University of Applied Sciences\\
E-Mail: grgr3403@hochschule-trier.de\\
Themensteller und Betreuer: Prof. Dr. Jürgen Graf\\
Cognitive Systems and Robotics}}

\maketitle


\begin{abstract}
Lorem ipsum dolor sit amet, consetetur sadipscing elitr, sed diam nonumy eirmod tempor invidunt ut labore et dolore magna aliquyam erat, sed diam voluptua.
Test \cite{Olcay.2020}
\end{abstract}

\section{Introduction}
Multi-Roboter systems, such as autonomous robots collaborating in a swarm, offer applications in unknown environments like mapping, exploration, or rescue missions [1]. To ensure collision-free paths, this work addresses collective navigation under limited sensor and communication ranges. To enhance system robustness, a nature-inspired swarm behavior is applied, based on three simple rules: cohesion, separation, and alignment [4]. While global motion planning assumes an initial map with static obstacles, this work focuses on local motion planning that can dynamically respond to environmental changes. Instead of Simultaneous Localization And Mapping (SLAM), a decentralized, potential field-based navigation approach (virtual force field in which robots are attracted or repelled) is used, relying solely on local sensor data and communication with neighboring robots. This enables the robots to make early decisions to avoid collisions while ensuring the system's scalability and flexibility.

\section{Theoretical background}
Im betrachteten Multi-Roboter-System werden die Roboter als Punktmassen in einer zweidimensionalen Umgebung modelliert. Die Systemdynamik basiert auf potenzialfeldbasierten Kräften, wobei nur die Position und Geschwindigkeit jedes Roboters betrachtet werden. Die Bewegungsdynamik jedes Roboters $i$ wird durch folgende Differentialgleichungen beschrieben:
\begin{equation}
    \dot{p}_i = v_i
\end{equation}
\begin{equation}
    \dot{v}_i = u_i
\end{equation}
Die Position des Roboters ändert sich also mit seiner Geschwindigkeit und diese wiederum durch eine Steuerkraft.
In der zeitlich diskreten Umsetzung folgt die Aktualisierung schrittweise über:
\begin{equation}
p_i(t_{k+1}) = p_i(t_k) + v_i \Delta t, \quad v_i(t_{k+1}) = v_i(t_k) + u_i \Delta t
\end{equation}
Jeder Roboter kann nur mit benachbarten Robotern innerhalb eines festen Kommunikationsradius $r_c$ bidirektional interagieren:
\begin{equation}
\mathcal{N}_i^\alpha = \{ j \in \mathcal{V} \; | \; \| p_j - p_i \| < r_c \}
\end{equation}
Um Schwarmverhalten zu erzeugen, werden zwischen zwei Robotern Distanzwerte berechnet, welche Abweichungsenergie, den Grad der Interaktion und Verbundenheit bestimmen (Gleichungen (4) bis (6)).

Die Steuereingabe $u_i$ eines Roboters $i$ wird über die Gleichung
\begin{equation}
u_i = u_i^\alpha + u_i^\beta + u_i^\gamma
\end{equation}
berechnet, wobei $u_i^\alpha$ für Schwarmverhalten, $u_i^\beta$ für Hindernisvermeidung und $u_i^\gamma$ für Navigation sorgt. Das Schwarmverhalten $u_i^\alpha$ setzt sich aus den Gleichungen 8 bis 10 zusammen, indem es Abstands- und Geschwindigkeitsanpassungen zu Nachbarn (8) sowie deren Wirkstärke in Abhängigkeit vom Abstand (9, 10) festlegt. 

Die Hindernisvermeidung $u_i^\beta$ setzt sich aus den Gleichungen 11 bis 13 zusammen, die definieren, welche Hindernispunkte ein Roboter erkennt (11), wie stark er von ihnen abgestoßen wird (12) und wie diese Abstoßungskraft in Abhängigkeit vom Abstand berechnet wird (13). Die Steuerkraft für die Navigation $u_i^\gamma$ berechnet sich mit Gleichung 14, um den Roboter zum Ziel zu führen.


\subsection{The problem statement}
Viele bestehende Navigationsansätze für Multi-Roboter-Systeme basieren auf künstlichen Potenzialfeldern, bei denen Roboter von Zielpunkten angezogen und von Hindernissen abgestoßen werden. Ein zentrales Problem dieser Methoden sind jedoch lokale Minima, in denen sich Anziehungs- und Abstoßungskräfte ausgleichen und Roboter feststecken. Besonders in Umgebungen mit konkaven oder labyrinthartigen Strukturen können herkömmliche Verfahren nicht zuverlässig entkommen und alternative Ansätze wie kamerabasierte Hinderniserkennung stoßen unter schwierigen Umweltbedingungen an ihre Grenzen. Für Einzelroboter existieren bereits verschiedene Strategien, um lokale Minima zu überwinden, deren Erfolg hängt jedoch stark von der Hinderniskonfiguration ab. Die Herausforderung für Multi-Roboter-Systeme liegt darin, kollisionsfrei und effizient in unbekannten und komplexen Umgebungen zu navigieren, trotz begrenzter Sensor- und Kommunikationsreichweiten. Hierzu sind kooperative und dezentrale Entscheidungsmechanismen erforderlich, um spontane und sichere Gruppenmanöver zu ermöglichen. Für diese Studie gingen die Autoren von keinem Rauschen (noise) aus.


\section{Single robot navigation}
Bevor ein Roboter im Schwarm handeln kann, muss er allein zurechtkommen. Dazu 
werden in diesem Abschnitt drei Methoden vorgestellt:
\begin{itemize}
    \item Tangential Navigation, inspiriert von [32],
    \item Corner avoidance und
    \item Motion planning at obstacle extemities.
\end{itemize}
Grundsätzlich möchte sich der Roboter zum \textit{desired goal} bewegen und achtet 
währenddessen in einem bestimmten Sensorradius \(r_{\mathrm{tan}}\) auf Hindernisse. Nimmt 
er dabei ein Hindernis (repräsentiert durch einen Punkt \( \hat{p}_{i,k} \)) wahr, weicht 
er unter Berücksichtigung der bewegungsrichtungsabhängigen Winkel \( \alpha_i \) und 
\( \beta_i \) und sich daraus ergebenen Rotationswinkel \( \gamma_i \), die die Ausweichrichtung 
(Gleichungen 15 bis 17) und ein virtuelles Ziel (18) bestimmen, parallel zur 
Hinderniskante, also tangential aus. Über eine angepasste Steuerkraft 
(19 ähnlich zu 14) wird der Roboter dabei zum \textit{virtual goal} beschleunigt.
Nimmt der Roboter zwei Hindernisse (repräsentiert durch zwei Punkte \( \hat{p}_{i,n} \) und \( \hat{p}_{i,n90} \)) 
wahr, die eine Ecke bilden, wendet er das \textit{Corner Avoidance} Manöver an. 
Dabei wird der Rotationswinkel \( \gamma_i \) um einen zusätzlichen Winkel \( \varepsilon_i \) 
ergänzt, der sich aus den Tangentenrichtungen zu beiden Hindernispunkten 
relativ zum Roboter berechnet (20, 21). Über Gleichung 22 wird anschließend 
ein virtuelles Ziel bestimmt, das den Roboter aus der Ecke herausführt.
Wenn der Roboter das Ende eines Hindernisses erreicht hat, führt er eine 
kreisförmige Bewegung um das Ende herum aus. Dazu merkt er sich den letzten 
Hindernispunkt \( \hat{p}_{i,e} \), der in seinem Sensorradius registriert wurde. Anhand der 
Gleichung 23 und dem Winkel \( \beta_{i,e} \), welcher sich durch die tangentiale Navigation 
ergab, berechnet der Roboter in gleichen Abständen virtuelle Ziele (berechnet 
durch Gleichung 24 und repräsentiert durch die Punkte \( p_{i,v1} \) bis \( p_d \)) und rotiert um einen 
festen Winkel \( \delta \), bis der Winkel zwischen Bewegungs- und Zielvektor 
(\textit{desired goal}) kleiner gleich \( \delta \) ist (25).


\section{The proposed navigation approach for a multi-robot system}
Der vorgestellte Ansatz erweitert den Navigationsalgorithmus für einzelne Roboter 
um eine Kommunikationsschnittstelle für die kollektive Bewegungsplanung mehrerer 
Roboter. Der Zusammenhalt der Gruppe könnte gefährdet werden, wenn jeder Roboter 
sein eigenes virtuelles Ziel verfolgt. Daher werden Informationen über virtuelle 
Ziele und kritische Punkte (wie in Fig. 4 dargestellt) untereinander ausgetauscht. 
Diese Informationen werden nach Relevanz, Aktualität und dem Prinzip "Erkennung vor 
Kommunikation" priorisiert (erkannte Hindernisse haben dabei Vorrang vor übermittelten 
Informationen aus dem Kommunikationsnetzwerk).
Das Kommunikationsnetzwerk besteht aus Informationspaketen, die sich in drei Typen 
gliedern: Orientierung, Endpunkte und Ecken.

\textbf{Orientierung (Fig. 5):} Enthält den Winkel \( \theta_i \) zum aktuellen Ziel, den erkannten 
Punkt \( \hat{p}_{i,k} \) auf dem Hindernis, den aktuellen Status des Roboter (Table 1) sowie den 
Zeitpunkt der Hinderniserkennung.

\textbf{Endpunkte (Fig. 6(a)):} Besteht aus dem zuletzt erkannten Punkt \( \hat{p}_{i,e} \), dem Winkel \( \omega_{i,e} \) 
zwischen dem Normalenvektor vom Roboter zum Hindernis und der x-Achse des inertialen 
Koordinatensystems sowie dem Zielwinkel \( \theta_{i,e} \) im Moment der Endpunkterkennung.

\textbf{Ecken (Fig. 6(b)):} Umfasst den Zielwinkel beim Eintritt (\( \theta_{i,\mathrm{ent}} \)) und Austritt 
(\( \theta_{i,\mathrm{ex}} \)) aus der Ecke sowie den Eckpunkt \( \hat{p}_{i,c} \), der als Schnittpunkt zweier 
Linien berechnet wird.

Bevor ein Roboter eine Aktion anhand der gegebenen Orientierungsinformationen aus 
dem Kommunikationsnetzwerk ausführt, berechnet er eine Relevanz-Funktion \( \mathrm{rel} \), 
welche im Intervall \( ]-\infty, 10] \) liegt. Dabei steht \( \mathrm{rel}=10 \) für eine 
sehr relevante und \( \mathrm{rel} \leq 0 \) für eine zu ignorierende Information. 
Unterschieden wird dabei wie folgt:
\begin{itemize}
    \item \textbf{Alter der Information:} Die Gleichung
    \begin{equation}
    \mathrm{rel}_t = 10 - \frac{10 \cdot (t_k - \hat{t}^c)}{d_t}
    \end{equation}
    bestimmt: je älter die Information, desto unwichtiger ist sie, mit \( t_k \)
    gleich aktuelle Zeit, \( \hat{t}^c \) gleich Registrierungszeit der Information von einem 
    anderen Roboter und \( d_t \in \mathbb{R} \) eine Konstante, die die Dauer 
    beschreibt, in der Information positive Relevanz haben kann.
    
    \item \textbf{Distanz vom Hindernis:} Sollte das Hindernis in Bewegungsrichtung liegen, 
    so ermittelt sich \( \mathrm{rel}_{\mathrm{dist}} \) mit 
    \begin{equation}
    \mathrm{rel}_{\mathrm{dist}} = 10 - \frac{10 \cdot \| \hat{p}^c - p_i \|}{d_x}
    \end{equation}
    mit \( d_x \in \mathbb{R} \) die maximale Distanz, die durch \( \| \hat{p}^c - p_i \| \) für eine positive 
    Relevanz erforderlich ist. Liegt es dahinter, dann mit
    \begin{equation}
    \mathrm{rel}_{\mathrm{dist}} = - \frac{10 \cdot \| \hat{p}^c - p_i \|}{d_x}
    \end{equation}
    und sollten keine Orientierungsinformationen gegeben sein, so 
    ist \( \mathrm{rel}_{\mathrm{dist}} = 0 \).
    
    \item \textbf{Relevanz der Orientierung basierend auf dem vorherigen Zeitschritt:} Jeder 
    Roboter erwartet für jeden Zeitschritt nur kleine Änderungen in der tangentialen 
    Navigation. Die Gleichung 
    \begin{equation}
    \mathrm{rel}_{\mathrm{exp}} = 10 - \frac{10 \cdot | \theta_i(t_k) - \theta^c |}{d_\theta}
    \end{equation}
    beschreibt die Relevanz der erwarteten Orientierung, wobei 
    \( |\theta_i(t_k) - \theta^c| \) der Vergleich mit der aktuellen Zielorientierung und 
    der neuen Orientierung ist und \( d_\theta \) die maximal erlaubte Winkeldifferenz für 
    eine positive Relevanz ist.
    
    \item \textbf{Auswertung des Senders:}
    \begin{equation}
    \mathrm{rel}_o =
    \begin{cases}
    10, & \text{wenn direkt gesendet}\\
    0, & \text{wenn nur weitergeleitet}
    \end{cases}
    \end{equation}
    
    \item \textbf{Auswertung anhand des Status:}
    \begin{equation}
    \mathrm{rel}_{\mathrm{type}} =
    \begin{cases}
    10, & \text{falls Status } c = 4 \text{ oder } 3 \\
    5,  & \text{falls Status } c = 1 \\
    0,  & \text{sonst}
    \end{cases}
    \end{equation}
\end{itemize}

Die endgültige Relevanz ergibt sich aus der Gleichung
\begin{equation}
\mathrm{rel}_n = \frac{c_{\mathrm{type}} \cdot \mathrm{rel}_{\mathrm{type}} + c_o \cdot \mathrm{rel}_o + c_{\mathrm{exp}} \cdot \mathrm{rel}_{\mathrm{exp}} + c_{\mathrm{dist}} \cdot \mathrm{rel}_{\mathrm{dist}} + c_t \cdot \mathrm{rel}_t}{c_{\mathrm{type}} + c_o + c_{\mathrm{exp}} + c_{\mathrm{dist}} + c_t}
\end{equation}
wobei die maximale Relevanz durch 
\begin{equation}
\mathrm{rel}_{\mathrm{max}} = \underset{\mathrm{rel}_n \in \mathcal{R}_i}{\mathrm{argmax}}(\mathrm{rel}_n)
\end{equation}
berechnet wird.

Ein Informationspaket ist für einen Roboter schließlich relevant, wenn die Bedingung
\begin{equation}
\mathrm{rel}_n \geq 0.95 \cdot \mathrm{rel}_{\mathrm{max}}
\end{equation}
erfüllt ist und passt daraufhin seine 
Bewegungsrichtung für den nächsten Zeitschritt an. Sollten mehrere Informationspakete 
diese Bedingung erfüllen, wird der Mittelwert daraus bestimmt.

Weiterhin prüft ein Roboter seine Distanz zu den übermittelten Hindernispunkten mit 
\begin{equation}
d_s \leq \| p_i - \hat{p}_e^c \| \leq R_{\mathrm{rel}}
\end{equation}
und sieht diese Punkte dann als seine eigenen Hindernispunkte an, sollte 
sie innerhalb der maximalen Distanz \( R_{\mathrm{rel}} \in \mathbb{R} \) liegen.


\section{Collective navigation using shared information}
In diesem Abschnitt wird erklärt, wie die Verfahren eines einzelnen Roboters auf ein 
Multi-Roboter-System adaptiert werden können. Dazu sind 7 Status definiert, in dem 
sich ein Roboter befinden kann, wobei sich jeder zum Start im Status 0 befindet 
(Erklärungen und Abhängigkeiten in Appendix A).

\subsection{Collective tangential navigation}
Die kollektive tangentiale Navigation umfasst die Status 1, 4 und 5. Erkennt ein 
Roboter ein Hindernis innerhalb seiner Wahrnehmungsreichweite \( r_{\mathrm{tan}} \), so leitet er, 
wie beim Einzelroboter, die tangentiale Navigation ein. Diese Aktion wird als 
Status 1 bezeichnet. Die dabei ermittelten Informationen werden an benachbarte 
Roboter weitergeleitet (Algorithmus 2).
Da die Steuergröße \( u_i^\alpha \) gemäß Gleichung 8 dazu führen kann, dass sich Roboter 
durch Abstoßung senkrecht zum Hindernis voneinander entfernen und so unbeabsichtigt 
ein Wechsel von Status 0 auf 1 ausgelöst werden könnte, wird eine Ignoranz-Bedingung 
(Bedingung 36) eingeführt. Dadurch können Roboter Hindernisse ignorieren.
Wird ein Roboter durch neue Informationen aus dem Kommunikationsnetzwerk in eine 
ungünstige Ausrichtung gebracht und erfüllt dabei Bedingung 37, so wechselt er in 
den Status 4 (Algorithmus 5), um seine Ausrichtung zu korrigieren.
Berechnet ein Roboter aufgrund empfangener Informationen ein virtuelles Ziel mittels 
Gleichung 38, befindet er sich im Status 5 (Algorithmus 6).
Um zu verhindern, dass Roboter Hindernisendpunkte zu spät erkennen, wenn diese nur 
von anderen benachbarten Robotern erfasst wurden, wird eine Distanzabschätzung mit 
einer Fallunterscheidung (Bedingung 39) vorgenommen. Damit kann der Roboter anhand 
der Entfernung zum Hindernis oder zu einer projizierten Position seine 
Zielausrichtung rechtzeitig anpassen und so Verzögerungen im Bewegungsablauf vermeiden.

\subsection{Collective corner avoidance}
Nähert sich ein Roboter einer Ecke, kann er sie, wie in Abschnitt 3.2 beschrieben, 
umfahren und dabei zusätzlich Informationen aus dem Kommunikationsnetzwerk nutzen. 
Diese Aktion wird als Status 3 bezeichnet. Damit dabei keine möglichen Durchgänge 
(z. B. zwischen zwei Hindernissen) übersehen werden, prüft der Roboter zuvor die 
Bedingung 40. Nur wenn diese erfüllt ist, wird die Ecke als solche erkannt.
Anschließend definiert der Roboter ein neues virtuelles Ziel und wechselt in Status 4, 
um sich neu auszurichten. Gleichzeitig sendet er Informationen über die erkannte Ecke 
an die benachbarten Roboter.
Werden die Bedingungen 40 jedoch nicht erfüllt, verbleibt der Roboter zunächst im 
Wartezustand (Status 6). In diesem Zustand überwacht er das Kommunikationsnetzwerk 
seiner Nachbarn. Wird dabei eine Information mit Status 3 oder 4 als relevant 
eingestuft, übernimmt der Roboter sie und verlässt den Wartezustand. Gibt es 
stattdessen relevante Informationen mit Status 1 oder 2, passt er sich entsprechend 
an und folgt entweder der Tangentialnavigation oder dem Manöver an einem Hindernisendpunkt.

\subsection{Collective motion at obstacle extremities}
Erkennt ein Roboter den Endpunkt eines Hindernisses, folgt er virtuellen Zielpositionen 
auf einem kreisförmigen Pfad (Status 2), wie in Abschnitt 3.3 beschrieben. Jeder 
Roboter legt individuelle Ziele auf diesem Pfad fest und teilt Informationen zum 
Endpunkt in seinem Netzwerk. Nach Erhalt der Endpunktdaten prüft ein Roboter seine 
Position durch orthogonale Projektion auf eine virtuelle Startlinie (Fig. 11(a)), 
bevor er die Kreisbewegung beginnt. Sollte der Roboter diese Startlinie noch nicht 
erreicht haben (Bedingung 42), setzt er weitere tangentiale Ziele und passt seine 
Geschwindigkeit an die Mindestgeschwindigkeit für die Kreisbewegung an 
(Gleichung 43 und 44). Erreicht er die Startlinie, beginnt die Kreisbewegung 
entsprechend des in Abschnitt 3.3 vorgestellten Schemas.
Während der Bewegung stellt eine Komponente aus Gleichung 8 sicher, dass der Roboter 
sein Ziel auf dem kreisförmigen Pfad erreicht. Überschreitet ein Roboter sein 
virtuelles Ziel (geprüft über eine Winkelbedingung), wird eine neue Zielposition 
berechnet. Die Roboter bewegen sich auf individuellen Kreisbahnen mit 
unterschiedlichen Radien, wodurch innere Kreise kürzere Wege bedeuten. 
Der Geschwindigkeitskonsens basiert bei der Kreisbewegung auf gleichen 
Winkelgeschwindigkeiten. Dafür wird eine minimale Winkelgeschwindigkeit (45) 
festgelegt und die Steuergröße aus Gleichung 8 für eine angepasste Steuergröße 
aus Gleichung 46 und 47 ersetzt. Dadurch können die Roboter gleichförmig das 
Hindernisende umkreisen.

\subsection*{Conditional Braking}

Wenn mehrere Roboter einer Gruppe eine kreisförmige Bewegung um ein Hindernis ausführen, können durch unterschiedliche Zeitpunkte beim Verlassen der Kreisbahn Geschwindigkeitsunterschiede entstehen, die zum Auseinanderreißen der Gruppe führen. Um dies zu verhindern, werden Roboter, die das Hindernis bereits passiert haben, abgebremst. Dieses Abbremsen bleibt solange aktiv, bis der letzte Roboter der Gruppe die Kreisbahn verlassen hat.

Zur Bestimmung dieses letzten Roboters wird ein neues Koordinatensystem definiert, dessen Ursprung im Endpunkt des Hindernisses liegt. In diesem System wird die Position jedes Roboters auf die x-Achse projiziert. Anhand dieser projizierten Positionen und über Gleichung 53 betrachten die Roboter ihre Nachbarschaft und identifizieren den Roboter, der am weitesten hinten liegt.

Über eine Gossip-ähnliche\footnote{lokaler, dezentraler und iterativer Informationsaustausch} Kommunikation wird die gesamte Gruppe über diesen Kandidaten informiert, bis alle Roboter den globalen hintersten Roboter ermittelt haben. Sobald dieser Roboter seine Kreisbewegung beendet hat, sendet er ein Reset-Signal, um das Bremsen der restlichen Roboter aufzuheben und die Gruppenbewegung wieder zu synchronisieren.

\subsection*{Watchdog timer}
Fällt die Geschwindigkeit eines Roboters unter einen definierten Schwellenwert, aktiviert er einen selbstüberwachenden Mechanismus, den \textit{Watchdog Timer}. Dabei speichert der Roboter seine aktuelle Position und den aktuellen Zeitpunkt. Der Timer wird wieder deaktiviert, sobald sich der Roboter innerhalb der folgenden Zeitschritte um mehr als einen festgelegten Schwellenwert bewegt (Roboter nimmt an einen möglichen Deadlock (ausweglose Situation) verlassen zu haben). Bewegt sich der Roboter jedoch über einen Zeitraum von 15 Sekunden kaum und wurde der Timer nicht deaktiviert, wechselt er automatisch in Status 0 und bewegt sich in Richtung desired goal. Diese Maßnahme kann zwar zu einer Fragmentierung der Gruppe führen, stellt jedoch sicher, dass jeder Roboter das desired goal erreicht.


\section{Simulation results}
Die Simulationsergebnisse werden anhand von zwei Szenarien dargestellt: 
einem mit einem Zickzack-Hindernis und einem mit einem Korridor mit Hindernissen. 
In beiden Fällen werden zwölf Roboter zufällig 
innerhalb eines Startbereichs positioniert, jeweils ohne Anfangsgeschwindigkeit. 
Zusätzlich wird ein gewünschtes Ziel (als rotes Kreuz markiert) definiert.

Die Roboter werden in den Abbildungen als schwarze Dreiecke dargestellt. 
Ihre Bewegung wird durch eine bunte Linie hinter den Dreiecken visualisiert, 
die Bewegungsrichtung entspricht der Ausrichtung der Dreiecksspitze. Die schwarzen 
Linien zwischen den Robotern zeigen das aktive Kommunikationsnetzwerk an. 
Die Abbildungen geben Momentaufnahmen zu bestimmten Zeitpunkten bzw. 
bei relevanten Ereignissen während der Simulation wieder.

Im ersten Szenario (Abbildung 13) zeigen sich folgende Abläufe:
\begin{itemize}
\item (a): Erkennung eines Hindernisses und Beginn der tangentialen Navigation
\item (b): Erkennung eines Eckpunktes und Ausführung des \textit{corner avoidance maneuvers}
\item (c) und (d): Erkennung eines Endpunktes und Einleitung der kreisförmigen Bewegung um das Hindernis
\item (e): Annäherung der Roboter an das Hindernis zur Optimierung der Navigation mithilfe von Gleichung 39
\item (f): Erreichen des gewünschten Zielpunkts
\end{itemize}

Im zweiten Szenario (Abbildung 15) kommen dieselben Verfahren zum Einsatz. 
Der Unterschied liegt lediglich darin, dass die Roboter hier durch einen Korridor mit Hindernissen navigieren müssen.

\subsection*{Guideline for parameter choice}
Die Parameter der Relevanzfunktion (Gleichung 32) sollten in Abhängigkeit des 
jeweiligen Szenarios gewählt werden. Ein globales Verhalten des 
Multi-Roboter-Systems, bei dem alle Roboter nahezu gleichzeitig identische 
Aktionen ausführen, ist dabei für kleinere Systeme mit sechs bis zwölf 
Robotern geeignet, wenn deren Verteilung im Verhältnis zu den Hindernissen 
gering ist. In diesem Fall sollten die Relevanz der Zeit des Informationspakets 
(\texttt{relt}) und der Relevanz des Aktionsstatus (\texttt{reltype}) höher 
gewichtet werden als die übrigen Werte, damit Roboter gleichartig und möglichst
synchron auf aktuelle Ereignisse im Kommunikationsnetzwerk reagieren.

Bei steigender Anzahl von Robotern oder einer dichteren Hinderniskonstellation 
kann ein solches globales Verhalten jedoch problematisch sein, da einzelne 
Roboter gleichzeitig mehrere relevante Informationspakete erhalten könnten. 
In diesen Situationen sollten Roboter stärker auf ihre eigene Position und lokale 
Informationen achten. Dazu werden die Relevanz des Abstands zum Hindernis 
(\texttt{reldist}), die Relevanz der erwarteten Orientierung für die nächste Aktion 
(\texttt{relexp}) sowie die Relevanz des Senders der Information (\texttt{relo}) 
höher gewichtet als \texttt{relt} und \texttt{reltype}.

Die Effekte dieser unterschiedlichen Gewichtungen zeigen sich in den folgenden Abbildungen:
\begin{itemize}
\item Abbildung 16 zeigt ein \textit{squeezing maneuver}, bei dem sich die Robotergruppe zwischen zwei kreisförmigen Hindernissen hindurch bewegt.
\item In Abbildung 17 ist zu erkennen, wie ein halbkreisförmiges Hindernis die Gruppe aufteilt. Die Roboter korrigieren daraufhin ihre Bewegungsrichtung, indem sie ihren aktuellen Aktionsstatus überprüfen und anpassen.
\item Abbildung 18 veranschaulicht, dass durch die priorisierte Hindernisvermeidung eine Fragmentierung der Gruppe auftritt. Dennoch erreichen letztlich alle Roboter das gewünschte Ziel.
\end{itemize}


\section{Conclusion}
In dieser Arbeit wurde ein kollektives Navigationskonzept für mehrere autonome 
Roboter ohne Vorwissen über die Umgebung vorgestellt. Es basiert auf 
einer tangentialen Fluchtstrategie und dem Austausch lokaler Informationen über 
ein Kommunikationsnetzwerk. Künstliche Kräfte aus Potentialfeldern ermöglichen 
kollisionsfreie, kooperative Manöver und ein schwarmähnliches Verhalten. Durch 
Optimierung der Parameter der Relevanzfunktion und Gewichtungsfaktoren kann der 
Fragmentierung von Robotergruppen entgegengewirkt werden. Sollten Roboter ihre Verbindung zu Anderen verlieren,
können sie auch allein ihr Ziel erreichen. Für holonomische\footnote{Position kann unabhängig der Ausrichtung geändert werden} Roboter ist dabei ein 
ausreichend großer Sicherheitsabstand und niedrige Geschwindigkeit notwendig. 
Künftige Arbeiten sollen sich auf eine Echtzeit-Umsetzung, verbesserte drahtlose 
Kommunikation und alternative Netzwerktopologien für Roboter konzentrieren.


\bibliographystyle{IEEEtran}
\bibliography{IEEEabrv,literature}

\end{document}
