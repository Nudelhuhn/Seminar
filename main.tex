\documentclass[conference]{IEEEtran}
\usepackage[utf8]{inputenc}
\usepackage{url}
\usepackage{amsmath}
\usepackage{amssymb}
\usepackage[colorlinks=true, linkcolor=blue, citecolor=blue, urlcolor=blue]{hyperref}

\begin{document}

\title{Seminar - Collective navigation of a multi-robot system in an unknown environment}
\author{\IEEEauthorblockN{Gregor Germerodt}
\IEEEauthorblockA{Hochschule Trier - Trier University of Applied Sciences\\
E-Mail: grgr3403@hochschule-trier.de\\
Themensteller und Betreuer: Prof. Dr. Jürgen Graf\\
Cognitive Systems and Robotics}}

\maketitle


\begin{abstract}
Lorem ipsum dolor sit amet, consetetur sadipscing elitr, sed diam nonumy eirmod tempor invidunt ut labore et dolore magna aliquyam erat, sed diam voluptua.
\end{abstract}


\section{Introduction}
Lorem ipsum dolor sit amet, consetetur sadipscing elitr, sed diam nonumy eirmod tempor invidunt ut labore et dolore magna aliquyam erat, sed diam voluptua.


\section{Related Work}
Multi-Agenten-Systeme wie autonome und im Schwarm zusammenarbeitende Roboter, bringen in unbekannten Umgebungen Möglichkeiten wie 
Kartierung, Erkundung oder Rettungsmissionen [1]. Um den Robotern dabei kollisionsfreie Wege zu gewährleisten, behandelt diese Arbeit 
kollektive Navigation mit limitierter Sensorik und Kommunikationsfähigkeiten.
Zur Robustheit des Systems, trägt das von der Natur abgesehene Schwarmverhalten einen Beitrag, welches hier in drei Regeln definiert 
wurde: Cohesion, separation and alignment [4].
Neben der globalen Bewegungsplanung, welche dem Roboter eine anfängliche Karte der Umgebung mit allen statischen Objekten bereitstellt, 
wird hier die lokale Bewegungsplanung behandelt, welche auf eine sich ändernde Umgebung automatisch reagieren kann.
Obwohl Simultaneous Localization And Mapping (SLAM) Methoden als herkömmliche lokale Navigationsstrategie erwähnt werden, wird hier 
stattdessen eine dezentrale, potenzialfeldbasierte Navigation mit lokaler Sensorik und Nachbarschaftskommunikation genutzt. Das hilft 
Robotern frühzeitig Entscheidungen zu treffen, zur Kollisionsvermeidung und die Systemgröße flexibel und skalierbar zu halten. 
\[
\mathcal{N}_i^{\alpha} = \left\{ j \in \mathcal{V} : \left\| \mathbf{p}_j - \mathbf{p}_i \right\| < r_c \right\}
\]


\section{Methods}
\cite{Olcay.2020} The authors present a decentralized navigation framework where each robot operates as a point mass in a 2D plane, 
controlled by a set of differential equations for position and velocity. The control input is composed of three main components: one 
for maintaining swarm-like flocking behavior, one for obstacle avoidance and one for navigation towards a shared goal position. To 
model the swarm interactions, each robot detects neighboring robots within a defined communication radius and interacts with them based 
on potential functions that encode attraction, repulsion, and alignment forces. To ensure smooth behavior, a sigma-norm is used to avoid 
singularities in the potential field calculations.

For obstacle avoidance, each robot uses range sensors with a limited detection radius. Once an obstacle is detected within a certain 
distance, the robot applies repulsive forces to maintain a safe margin. To avoid local minima near concave obstacles, the authors 
implement a tangential navigation strategy: when a robot detects an obstacle, it computes a virtual goal position tangentially offset 
from the direct path to the goal. This allows the robot to follow the obstacle's boundary while gradually progressing toward the goal.

When navigating around corners, if a robot detects two distinct obstacle points simultaneously, it identifies the presence of a concave 
corner. The robot then determines a new temporary virtual goal, strategically placed to facilitate a smooth turn. Similarly, for obstacle 
endpoints where the robot loses contact with the boundary, it initiates a controlled circular motion around the endpoint based on the 
last known obstacle point, maintaining group cohesion and avoiding local minima.

To extend this single-robot navigation framework to multi-robot systems, a communication protocol is introduced, allowing robots to 
share information about detected obstacles, endpoints, and corners. Each robot maintains a status variable indicating its current 
behavior (e.g., moving to goal, tangential navigation, corner avoidance) and uses a relevance function to prioritize received 
information. This function takes into account the information's age, spatial proximity, and the sender's status. Only highly relevant 
information influences the robot's decision-making, enabling decentralized, coordinated group maneuvers while preserving individual autonomy.

Additionally, mechanisms are implemented to handle group cohesion at obstacle endpoints through a conditional braking system, where 
the robot furthest behind in a circular maneuver signals the group to resume normal movement. A watchdog timer is also introduced to 
prevent deadlocks: if a robot's speed drops below a threshold for a prolonged period, it resets its navigation status to resume free movement.


\section{Analysis/Experiments/Proofs}
\cite{Olcay.2020} To evaluate the performance of their navigation framework, the authors conduct a series of simulations with varying 
group sizes and environmental layouts. In one scenario, 12 robots must traverse a zigzag-shaped obstacle. The simulation results 
demonstrate that the robots successfully avoid collisions, detect corners, and navigate obstacle endpoints through coordinated group 
maneuvers. The average speed of the swarm drops at critical points, such as obstacle extremities, indicating the robots' tendency to 
decelerate and maintain group cohesion while executing circular maneuvers.

A second scenario involves navigating through a narrow corridor with scattered obstacles. Here, the robots' ability to react to 
dynamic, locally perceived obstacles and coordinate with nearby agents enables them to maintain a consistent formation and avoid 
collisions despite the tight space.

The authors further test scalability by increasing the number of robots to 20 and introducing more complex environments, including 
two closely spaced circular obstacles, a semi-circular barrier, and numerous small obstacles scattered across the space. In these 
cases, the robots exhibit robust collective behavior. Although minor fragmentation occurs in very dense obstacle fields, individual 
robots continue to navigate safely towards the goal without collisions or deadlocks, demonstrating the system's resilience.

The simulations confirm that the relevance-based communication strategy effectively prevents conflicting maneuvers and enables 
decentralized consensus on navigation decisions. The conditional braking mechanism ensures that group members at the back can signal 
when it is safe for the swarm to resume normal speed, preserving cohesion during complex maneuvers. The watchdog timer successfully 
resolves potential deadlock situations without negatively affecting the overall mission success.

Overall, the experimental results support the authors' claim that their decentralized, potential field-based multi-robot navigation 
framework is effective for unknown, static environments. It balances local autonomy and collective coordination, handles concave 
obstacles and narrow passages, and remains scalable to moderately large robot groups.


\section{Conclusion}
Lorem ipsum dolor sit amet, consetetur sadipscing elitr, sed diam nonumy eirmod tempor invidunt ut labore et dolore magna aliquyam erat, sed diam voluptua.


\bibliographystyle{IEEEtran}
\bibliography{IEEEabrv,literature}

\end{document}
